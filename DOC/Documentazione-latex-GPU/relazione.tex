% !TeX spellcheck = <none>
\documentclass{article}
\usepackage{graphicx}
\usepackage{float}
\usepackage[italian]{babel}
\usepackage[latin1]{inputenc}
\usepackage{lipsum}% http://ctan.org/pkg/lipsum
\usepackage{textgreek}
\usepackage{listings}
\usepackage{xcolor}
\usepackage{hyperref}
\usepackage{subcaption}

\lstset{
	basicstyle=\small,
	columns=fullflexible,
	frame=single,
	breaklines=true,
	postbreak=\mbox{\textcolor{red}{$\hookrightarrow$}\space},
	emph={
		__shared__
	}
}

\begin{document}
	\title{Progetto di Gestione Dell'Informazione Geospaziale - DBSCAN}
	\author{Damiano~Bianda}
	\maketitle
\begin{abstract}
	%\boldmath\\
	In questo progetto si vuole implementare l'algoritmo di clustering DBSCAN tramite Java.\\
	Dato un insieme di coordinate proiettate, queste vengono classificate secondo il cluster d'appartenenza o come outliers.\\
	Infine i risultati ottenuti sono rappresentati attraverso una mappa utilizzando il software QGIS.
\end{abstract}
\section{Algoritmo}
\subsection{Introduzione}
DBSCAN � un algoritmo di clustering partitivo basato sulla densit�. I principali vantaggi sono che � possibile scoprire clusters di forme arbitrarie (dimensioni e forme differenti) ed identificare gli outliers, ossia i punti all'infuori di un area densamente popolata e quindi non appartenenti a nessun cluster. I svantaggi sono invece che essendo un algoritmo parametrico � necessario definire dei parametri in base al tipo di dati da analizzare e che alcuni dataset presentano il problema della densit� variabile, ossia sono presenti pi� cluster ma a densit� diverse e quindi dati dei parametri non � possibile in un esecuzione identificarli tutti.
\subsection{Definizioni}
DBSCAN � parametrizzato tramite \textepsilon\enspace e MinPoints.\\
Il vicinato di un punto sono tutti i punti che ricadono nel cerchio con centro pari alla sua coordinata e raggio \textepsilon.\\
Un punto � detto:
\begin{itemize}
	\item core point se il suo vicinato contiene almeno MinPoints elementi.
	\item border point se non � un core point, ma � nel vicinato di uno o pi� core point
	\item noise point se non � n� core point, n� border point
\end{itemize}
Le definizioni seguenti descrivono un rapporto di connessione tra i punti e servono per definire il concetto di cluster:
\begin{itemize}
	\item directly density-reachable\\
	un punto p � detto directly density-reachable da un punto q se p � nel vicinato di q e se q � un core point
	\item density-reachable\\
	un punto p � detto density-reachable da un punto q se c'� una serie di punti, in cui il primo � q e l'ultimo � p, dove ogni elemento � directly density-reachable dal precedente
	\item density-connected\\
	un punto p � detto density-connected ad un punto q se c'� un punto r tale che p e q sono density-reachable da r
\end{itemize}
Quindi un cluster � un insieme massimo di punti density-connected.
\subsection{Esecuzione di DBSCAN}
DBSCAN itera su tutti i punti presenti nel dataset e per ognuno si determina se � un core point.\\
In caso positivo si crea un cluster che viene espanso coi punti presenti nel suo vicinato.\\
Il processo si ripete iterativamente controllando i punti appena aggiunti, se a loro volta sono core point viene aggiunto il loro vicinato e cos� via fin quando tutti i punti density-reachable sono stati inglobati nel cluster.\\
Il vicinato non viene aggiunto al cluster quando un punto � border point, poich� non esistono punti directly density-reachable da questo.\\
Una volta terminato l'algoritmo si ottiene un dataset con i suoi elementi etichettati secondo i differenti cluster o come noise.

\pagebreak
\section{Implementazione}
\subsection{Pseudocodice}
\begin{lstlisting}
NOISE := 0

void initDataSet(dataset):
	foreach point in points:
		point.cluster := NOISE

bool isCorePoint(neighborhood, minPoints):
	return neighborhood.size() >= minPoints

List<Point> neighborhood(points , point, epsilon):
	neighbors = [ ]
	foreach candidateNeighbor in points:
		if distance(candidateNeighbor, point) <= epsilon:
			neighbors.add(candidateNeighbor)
	return neighbors

float distance(pointA, pointB):
	return sqrt((pointA.x - pointB.x)^2 + (pointA.y - pointB.y)^2))

void DBSCAN(points, epsilon, minPoints):
	initDataSet(points)
	clusterLabel := 1
	foreach point in points:
		if point.cluster = NOISE:
		neighbors := neighborhood(points, point, epsilon)
		if isCorePoint(neighborhood, minPoints):
			foreach neighbor in neighbors:
			neighbor.cluster := clusterLabel
			neighbors.remove(point)
			while neighbors.size() > 0:
				currentPoint := neighbors.getFirst()
				currentNeighborhood := neighborhood(points, currentPoint, epsilon)
				if isCorePoint(currentNeighborhood, currentPoint):
					foreach currentNeighbor in currentNeighborhood:
						if currentNeighbor.cluster := NOISE
							neighbor.add(currentNeighbor)
							neighbor.cluster := clusterLabel
			clusterLabel := clusterLabel + 1
\end{lstlisting}

Inizialmente ogni punto del dataset non appartiene a nessun cluster, quindi durante la lettura dei dati vengono classificati come NOISE.\\
Successivamente si itera su ogni punto, quelli che hanno ancora la label NOISE sono candidati a creare un cluster, quindi si controlla se rispettano le condizioni per essere identificati come border points, in caso negativo l'elemento rimane NOISE e potr� essere inglobato in un altro cluster durante l'esecuzione dell'algoritmo.\\
Se invece un punto risulta essere un core point, a questo punto � necessario espandere il cluster tramite un approccio breadth first.\\
Viene mantenuta una coda che inizialmente contiene solo i nodi directly density-reachable da quello di partenza, fino a quando la coda non � vuota si estrae il primo elemento ed il suo vicinato.\\
Se il nuovo elemento estratto risulta anch'esso essere un core point gli elementi del vicinato che non appartengono ancora a nessun cluster vengono assegnati al label corrente e inseriti nella coda.\\
Quando la coda � vuota significa che ogni elemento density-reachable a partire da quello iniziale � stato assegnato al cluster, quindi viene incrementato l'id globale che identifica il cluster corrente e si procede a iterare sui nodi ancora classificati come NOISE.\\

\subsection{Codice}
\begin{lstlisting}
public static void DBSCAN(ArrayList<Coordinate> points, float eps, int minPoints){
	int clusterLabel = Coordinate.NOISE + 1;
	for (Coordinate point: points){
		if(point.isNoise()){
			final ArrayList<Coordinate> queue = neighborhood(points, point, eps);
			if(queue.size() < minPoints) {
				continue;
			}
			for(Coordinate c: queue){
				c.setLabel(clusterLabel);
			}
			queue.remove(point);
			point.setLabel(clusterLabel); // credo che non serve
			while (!queue.isEmpty()){
				final Coordinate currentPoint = queue.remove(0);
				final ArrayList<Coordinate> currentNeighborhood = neighborhood(points, currentPoint, eps);
				if (currentNeighborhood.size() >= minPoints){
					for(Coordinate currentNeighbor: currentNeighborhood){
						if(currentNeighbor.isNoise()){
							queue.add(currentNeighbor);
							currentNeighbor.setLabel(clusterLabel);
						}
					}
				}
			}
			clusterLabel++;
		}
	}
}
\end{lstlisting}

\end{document}
